\documentclass[fleqn,10pt]{wlscirep}
\usepackage[utf8]{inputenc}
\usepackage[T1]{fontenc}
\usepackage{url}
\usepackage{subfig}
\graphicspath{ {./images/} }
\title{Formação de padrões de vegetação através de autômatos celulares}

\author[1*]{Luiz Dutra}
\affil[1]{Departamento de Estatística e Informática
Universidade Federal Rural de Pernambuco – Recife, PE – Brasil}

\affil[*]{luiz.dutra@ufrpe.br}



\begin{abstract}
Padrões formados por vegetações de diversos tipos são um fenômeno importante para a sobrevivência destes organismos, mostrando uma capacidade de auto-organização vantajosa. Este trabalho propõe um modelo de autômato celular para analisar a formação e influência de variáveis no surgimento destes padrões, considerando características da vegetação, solo e clima no conjunto de regras do modelo. Os resultados do autômato celular mostram capacidade de replicar estruturas de vegetação semelhantes a exemplos reais, considerando diferentes relações entre características da vegetação e níveis de precipitação.
\end{abstract}
\keywords{Autômatos Celulares, Padrões de Vegetação, Auto-Organização, Padrões de Turing}


\begin{document}


\flushbottom
\maketitle
% * <john.hammersley@gmail.com> 2015-02-09T12:07:31.197Z:
%
%  Click the title above to edit the author information and abstract
%
\thispagestyle{empty}

\section*{Introdução}

Na natureza são observados diversos padrões de vegetação auto-organizada\cite{{doi:10.1086/342078}}. Esses padrões auto-organizados surgem através das relações entre os organismos \cite{doi.org/10.1111/oik.06373} e fatores externos como alterações climáticas \cite{MERON20161} e características do solo\cite{lejeune2002localized}.
	
\begin{figure}[ht]
\hfill
\begin{minipage}{0.45\linewidth}
\includegraphics[width=\linewidth]{Gapped_Bush_Niger_Nicolas_Barbier}
\caption {A imagem mostra uma vegetação do tipo Tiger Bush localizada na Nigéria com vãos em sua estrutura\cite{wiki:pat}.}
\label{fig:gapped}
\end{minipage}
\hfill
\begin{minipage}{0.45\linewidth}
\includegraphics[width=\linewidth]{labi}
\caption {A imagem mostra outra vegetação na Nigéria possuindo um padrão de labirinto com pontos\cite{doi:10.1086/342078}.}
\label{fig:lab}
\end{minipage}
\hfill
\end{figure}

A formação destes padrões é explorada através de abordagens matemáticas como o modelo de Reação-Difusão de Turing\cite{doi:10.1086/342078} e a Teoria do Controle Ótimo\cite{hou2025pattern}. Este trabalho tem como objetivo criar simulações de formação destes padrões através de um Autômato Celular, considerando fatores como a influência de organismos sobre outros, características do solo e principalmente níveis de precipitação.

\section*{Metodologia Computacional}

\subsection*{Modelagem}
Para estabelecer a simulação foi utilizado um Autômato Celular Estocástico Contínuo, com o tamanho de seu espaço definido em \emph{200x200}. Com o objetivo de simular o solo e a vegetação separadamente o autômato possui dois valores de estado. Os estados de vegetação e solo são definidos em valores reais e limitados a partir de parâmetros definidos no modelo, \emph{plant limit} e \emph{soil limit} respectivamente.
O estado da vegetação possui intervalo  $[0, \infty)$ onde o parâmetro limitador define quando uma célula de vegetação não pode utilizar do solo e aumentar seu valor, um valor de estado igual a zero representa um local sem vegetação, para cada célula de vegetação é denominado um valor de tempo de vida (\emph{lifetime}) em sua criação.
O estado do solo possui intervalo  $(\infty, \emph{soil limit}]$ onde o parâmetro limitador define o valor máximo de uma célula de solo, um valor de solo menor ou igual a zero representa um solo depletado.
Para definir um tipo de vegetação e clima específico, o modelo contém outros parâmetros que são utilizados no conjunto de regras do modelo.
\begin{itemize}
\item \emph{soil cost}: Custo de solo que denomina a quanto a vegetação utiliza do solo em uma iteração.
\item\emph{ne soil scale}: Define a escala de custo de solo para células vizinhas.
\item \emph{soil scale}: Valor base utilizado para realizar uma interpolação linear do \emph{soil cost} de acordo com o estado da vegetação.
\item \emph{min plant maturity, min soil}: Define o valor mínimo do estado da vegetação da célula e do solo vizinho, respectivamente, para reprodução.
\item \emph{rain interval}: Representa o intervalo de iterações entre precipitações.
\item \emph{rain amount}: Define a quantidade de precipitação de cada iteração.
\item\emph{plant decay, soil decay}: Define a degradação, referente à água, do solo, com e sem vegetação, respectivamente.
\item\emph{lifetime$\mu$, lifetime$\sigma$}: Utilizados na distribuição do tempo de vida das células.
\end{itemize}

\subsection*{Regras}
O comportamento de cada célula é definido por um conjunto de regras que são aplicadas em toda iteração através dos valores de estados da célula e células vizinhas, que são definidas como toda célula em proximidade imediata, incluindo diagonais. 

Para melhor entendimento, o estado da vegetação será representado como \emph{plantState} e o estado do solo como \emph{soilState}.
O conjunto de regras é definido em:

\begin{enumerate}
\item Toda célula onde \emph{plantState} > 0 e \emph{lifetime > 0}  irá absorver o \emph{soilState} de sua própria célula e de células vizinhas a partir das fórmulas \emph{f(t)}  e \emph{g(t)}, respectivamente:

\emph{let} $t = plantState/plantLimit$

$f(t) = soilCost * ((1 - t)*soilScale + t*1)$

$g(t) = f(t) * neSoilCost$

A função $f(t)$ aplica uma interpolação linear, com o objetivo de gerar um viés a favor de células com estado de vegetação maior, gerando competitividade. Na função $g(t)$ é aplicada a escala de custo para solos vizinhos, simulando o quão eficiente a vegetação é em utilizar do solo vizinho em comparação ao seu\cite{meron2019vegetation}.


\item Uma célula que possui vegetação mas o seu $lifetime == 0$, seu estado será somado ao estado de solo atual em conformidade com o limite do solo, com o objetivo de preservar a qualidade daquela região enquanto permite a mudança de valor e forma das células vizinhas.

\item A cada iteração é aplicado o \emph{soil decay} ou \emph{plant decay} de acordo com o estado da célula.

\item Após o número de iterações definido em \emph{rainInterval}, a precipitação do período é somada ao estado do solo seguindo:

 $soilState = min(soilState + rainInterval * rainAmount, soilLimit)$

\item A possibilidade de fertilização de um solo sem vegetação é definida por:

$plant state_0 >= minPlantMaturity$

$soil State_1 >= minSoil$ $and$ $plantState_1 == 0$

\item Quando um solo é fertilizado seu tempo de vida é definido de maneira pseudoaleatória por uma distribuição normal.

$\mathcal{N}(lifetime\mu, lifetime\sigma^2)$

Esse tempo de vida determinado de forma pseudoaleatória permite introduzir elementos de imprevisibilidade na auto-organização das células.


\end{enumerate}


\section*{Resultados}
As simulações foram realizadas com 10000 iterações cada, com uma população de 5 células com \emph{plant state} igual a 0.5 e \emph{lifetime} igual a 10, em posições aleatórias,  o valor do estado do solo inicial foi distribuído de maneira aleatória uniforme no intervalo $[0.4, 0.6]$,  utilizando a mesma seed do gerador de números pseudoaleatórios com diferentes níveis de decaimento do solo a fim de analisar a formação de diferentes padrões de vegetação de acordo com estas condições. Todas as vegetações de cada simulação possuem as mesmas características e decaimento de solo com vegetação de \emph{0.0002}, equivalente à quantidade de precipitação por iteração que possui um intervalo de 400 iterações entre cada precipitação, aplicando o valor acumulado ao solo, como definido na regra \emph{4}.

Após realizar as simulações os autômatos tiveram uma formação de padrões similares à vegetações naturais. As células alcançaram padrões como:  Vãos em sua estrutura (fig.\ref{fig:12}), conexões semelhante a um labirinto (fig.\ref{fig:25}) ou formação de ilhas de vegetação(fig.\ref{fig:04}). Também é possível observar a progressão da formação destes padrões, por exemplo, a formação da estrutura de labirinto a partir da expansão dos vãos, ao se conectarem.

\begin{figure}[ht]
	\hfill
	\begin{minipage}{0.45\linewidth}
		\includegraphics[width=1\linewidth]{../../results/0.0012}
		\caption {Simulação com decaimento atribuído em \emph{0.0012} com um padrão similar à fig.\ref{fig:gapped}.}
		\label{fig:12}
	\end{minipage}
	\hfill
	\begin{minipage}{0.45\linewidth}
		\includegraphics[width=1\linewidth]{../../results/0.0025}
		\caption {Simulação com um decaimento maior de \emph{0.0025} com uma formação de labirinto similar à fig.\ref{fig:lab}.}
		\label{fig:25}
	\end{minipage}
	\hfill
\end{figure}
\clearpage
\begin{figure}[ht]
	\hfill
	\begin{minipage}{0.45\linewidth}
		\includegraphics[width=1\linewidth]{../../results/0.004}
		\caption {Simulação com decaimento de \emph{0.004} com um padrão de pontos ou ilhas de vegetação.}
		\label{fig:04}
	\end{minipage}
	\hfill
	\begin{minipage}{0.45\linewidth}
		\includegraphics[width=1\linewidth]{../../results/0.0025_less}
		\caption {Simulação com uma modificação no parâmetro de \emph{plant limit} de \emph{1} para \emph{0.5}, com um decaimento de \emph{0.0025}, formando 					      uma vegetação esparsa mas rasa.}
		\label{fig:25less} 
	\end{minipage}
	\hfill
\end{figure}


\section*{Conclusão e Discussão}
 De acordo com os resultados do modelo, é observável os diferentes comportamentos de acordo com a parametrização, que busca refletir valores com equivalência no mundo real, aplicada.

O fator dominante para influência na formação de uma vegetação de mesmas características foi o decaimento do solo, que em valores maiores representa um clima mais árido com mais perda de água pelo solo; com o aumento deste é possível observar o surgimento de padrões que organizam a vegetação local de forma mais densa, sendo formações vantajosas devido ao potencial de preservar o valor de solo, principalmente influenciados pela regra 2, que permite o retorno de valor de uma vegetação ao solo, e 3, onde o valor de \emph{plant decay} e equivalente ao valor de precipitação, nas simulações, ou menor que o valor de \emph{soil decay}. É demonstrado na diferença dos resultados da fig.\ref{fig:25} e fig.\ref{fig:25less}, que a redução do valor máximo do estado da vegetação de uma célula, sem alterar características de solo ou precipitação, permite o maior uso de solo por células vizinhas, permitindo um crescimento de área mais expansivo mas com baixa densidade comparada aos outros padrões.

O uso deste modelo de Autômato Celular permitiu realizar simulações que se assemelham à realidade, com a possibilidade de modificar características da vegetação, solo e precipitação para definir condições específicas que influenciam no surgimento de padrões auto-organizados.

\bibliography{sample}

\section*{Apêndices}
\href{https://github.com/LuizzDutra/Cellular-Automata-UFRPE_2025.2}{GitHub} possuindo o código fonte, versionamento deste documento e arquivos das simulações

\end{document}