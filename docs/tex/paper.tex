\documentclass[fleqn,10pt]{wlscirep}
\usepackage[utf8]{inputenc}
\usepackage[T1]{fontenc}
\usepackage{url}
\graphicspath{ {./images/} }
\title{Formação de padrões de vegetação através de autômatos celulares}

\author[1*]{Luiz Dutra}
\affil[1]{Departamento de Estatística e Informática
Universidade Federal Rural de Pernambuco – Recife, PE – Brasil}

\affil[*]{luiz.dutra@ufrpe.br}

%\keywords{Keyword1, Keyword2, Keyword3}

\begin{abstract}
Example Abstract. Abstract must not include subheadings or citations. Example Abstract. Abstract must not include subheadings or citations. Example Abstract. Abstract must not include subheadings or citations. Example Abstract. Abstract must not include subheadings or citations. Example Abstract. Abstract must not include subheadings or citations. Example Abstract. Abstract must not include subheadings or citations. Example Abstract. Abstract must not include subheadings or citations. Example Abstract. Abstract must not include subheadings or citations.
\end{abstract}
\begin{document}

\flushbottom
\maketitle
% * <john.hammersley@gmail.com> 2015-02-09T12:07:31.197Z:
%
%  Click the title above to edit the author information and abstract
%
\thispagestyle{empty}

\noindent Please note: Abbreviations should be introduced at the first mention in the main text – no abbreviations lists. Suggested structure of main text (not enforced) is provided below.

\section*{Introdução}

Na natureza são observados diversos padrões de vegetação auto-organizada\cite{{doi:10.1086/342078}}. Esses padrões auto-organizados surgem através das relações entre os organismos \cite{doi.org/10.1111/oik.06373} e fatores externos como alterações climáticas \cite{MERON20161} e características do solo\cite{lejeune2002localized}.
	
\begin{figure}[ht]
\centering
\textbf{A}
\includegraphics[width=0.45\linewidth]{Gapped_Bush_Niger_Nicolas_Barbier}
\hfill
\textbf{B}
\includegraphics[width=0.45\linewidth]{labi}
\caption{A imagem \textbf{A} mostra uma vegetação do tipo Tiger Bush localizada na Nigéria com vãos em sua estrutura\cite{wiki:pat}. A imagem B mostra outra vegetação na Nigéria possuindo um padrão de labirinto com pontos\cite{doi:10.1086/342078}.}
\label{fig:gapped}
\end{figure}

A formação desses padrões é explorada através de abordagens matemáticas como o modelo de Reação-Difusão de Turing\cite{doi:10.1086/342078} e a Teoria do Controle Ótimo\cite{hou2025pattern}. Este trabalho tem como objetivo criar simulações de formação desses padrões através de um Autômato Celular, considerando fatores como a influência de organismos sobre outros, características do solo e principalmente níveis de precipitação.

\section*{Metodologia Computacional}

\subsection*{Modelagem}
Para estabelecer a simulação foi utilizado um Autômato Celular Estocástico Contínuo, com o tamanho de seu espaço definido em \emph{200x200}. Com o objetivo de simular o solo e a vegetação separadamente o autômato possui dois valores de estado. Os estados de vegetação e solo são definidos em valores reais e limitados a partir de parâmetros definidos no modelo, \emph{plant limit} e \emph{soil limit} respectivamente.
O estado da vegetação possui intervalo  $[0, \infty)$ onde o parâmetro limitador define quando uma célula de vegetação não pode utilizar do solo e aumentar seu valor, um valor de estado igual a zero representa um local sem vegetação, para cada célula de vegetação é denominado um valor de tempo de vida (\emph{lifetime}) em sua criação.
O estado do solo possui intervalo  $(\infty, \emph{soil limit}]$ onde o parâmetro limitador define o valor máximo de uma célula de solo, um valor de solo menor ou igual a zero representa um solo depletado, o valor do estado do solo inicial é distribuído de maneira pseudoaleatória uniforme no intervalo [0.4, 0.6].
Para definir um tipo de vegetação e clima específico, o modelo contém outros parâmetros que são utilizados no conjunto de regras do modelo.
\begin{itemize}
\item \emph{soil cost}: Custo de solo que denomina a quanto a vegetação utiliza do solo em uma iteração.
\item\emph{ne soil scale}: Define a escala de custo de solo para células vizinhas.
\item \emph{soil scale}: Valor base utilizado para realizar uma interpolação linear do \emph{soil cost} de acordo com o estado da vegetação.
\item \emph{min plant maturity, min soil}: Define o valor mínimo do estado da vegetação da célula e solo vizinho, respectivamente, para reprodução.
\item \emph{rain interval}: Representa o intervalo de iterações entre precipitações.
\item \emph{rain amount}: Define a quantidade de precipitação de cada iteração.
\item\emph{plant decay, soil decay}: Define a degradação do solo com e sem vegetação, respectivamente.
\item\emph{lifetime$\mu$, lifetime$\sigma$}: Utilizados na distribuição do tempo de vida das células.
\end{itemize}

\subsection*{Regras}
O comportamento de cada célula é definido por um conjunto de regras que são aplicadas em toda iteração através dos valores de estados da célula e células vizinhas, que são definidas como toda célula em proximidade imediata, incluindo diagonais. 

Para melhor entendimento o estado da vegetação será representado como \emph{plantState} e o estado do solo como \emph{soilState}.
O conjunto de regras é definido em:

\begin{itemize}
\item Toda célula onde \emph{plantState} > 0 e \emph{lifetime > 0}  irá absorver o \emph{soilState} de sua própria célula e de células vizinhas a partir das fórmulas \emph{f(t)}  e \emph{g(t)}, respectivamente:

\emph{let} $t = plantState/plantLimit$

$f(t) = soilCost * ((1 - t)*soilScale + t*1)$

$g(t) = f(t) * neSoilCost$

A função $f(t)$ aplica uma interpolação linear, com o objetivo de gerar um viés a favor de células com estado de vegetação maior, gerando competitividade.


\item Uma célula que possui vegetação mas o seu $lifetime == 0$, seu estado será somado ao estado de solo atual em conformidade com o limite do solo, com o objetivo de preservar a qualidade daquela região enquanto permite a mudança de valor e forma das células vizinhas.

\item A cada iteração é aplicado o \emph{soil decay} ou \emph{plant decay} de acordo com o estado da célula.

\item Após o número de iterações definido em \emph{rainInterval}, a precipitação do período é somada ao estado do solo seguindo $soilState = max(soilState + rainInterval * rainAmount, soilLimit)$

\item A possibilidade de fertilização de um solo sem vegetação é definida por 

$plant state_0 >= min plant maturity$

$soil State_1 >= min soil$ $and$ $plantState_1 == 0$

\item Quando um solo é fertilizado seu tempo de vida é definido de maneira pseudoaleatória por uma distribuição normal.

$\mathcal{N}(lifetime\mu, lifetime\sigma^2)$

Esse tempo de vida determinado de forma pseudoaleatória permite introduzir elementos de imprevisibilidade na auto organização das células.


\end{itemize}


\section*{Resultados}

Up to three levels of \textbf{subheading} are permitted. Subheadings should not be numbered.

\subsection*{Subsection}

Example text under a subsection. Bulleted lists may be used where appropriate, e.g.

\begin{itemize}
\item First item
\item Second item
\end{itemize}

\subsubsection*{Third-level section}
 
Topical subheadings are allowed.

\section*{Discussion}

The Discussion should be succinct and must not contain subheadings.



\bibliography{sample}

\noindent LaTeX formats citations and references automatically using the bibliography records in your .bib file, which you can edit via the project menu. Use the cite command for an inline citation, e.g.  \cite{Hao:gidmaps:2014}.

For data citations of datasets uploaded to e.g. \emph{figshare}, please use the \verb|howpublished| option in the bib entry to specify the platform and the link, as in the \verb|Hao:gidmaps:2014| example in the sample bibliography file.

\section*{Acknowledgements (not compulsory)}

Acknowledgements should be brief, and should not include thanks to anonymous referees and editors, or effusive comments. Grant or contribution numbers may be acknowledged.

\section*{Author contributions statement}

Must include all authors, identified by initials, for example:
A.A. conceived the experiment(s),  A.A. and B.A. conducted the experiment(s), C.A. and D.A. analysed the results.  All authors reviewed the manuscript. 

\section*{Additional information}

To include, in this order: \textbf{Accession codes} (where applicable); \textbf{Competing interests} (mandatory statement). 

The corresponding author is responsible for submitting a \href{http://www.nature.com/srep/policies/index.html#competing}{competing interests statement} on behalf of all authors of the paper. This statement must be included in the submitted article file.

\begin{table}[ht]
\centering
\begin{tabular}{|l|l|l|}
\hline
Condition & n & p \\
\hline
A & 5 & 0.1 \\
\hline
B & 10 & 0.01 \\
\hline
\end{tabular}
\caption{\label{tab:example}Legend (350 words max). Example legend text.}
\end{table}

Figures and tables can be referenced in LaTeX using the ref command, e.g. Figure \ref{fig:stream} and Table \ref{tab:example}.

\end{document}